\documentclass{article}
\usepackage{chemfig}
\renewcommand*\printatom[1]{\ensuremath{\mathsf{#1}}}

\title{Definitions - Organic Chemistry}
\author{G. Nordin}
\date{\today}

\begin{document}
\maketitle

\begin{center}
\begin{tabular}{ |c|c|c| } 
 \hline
 Number of Carbons & Parent chain name & Substituent Name \\ 
 \hline
 1 & methane & methyl \\ 
 2 & ethane & ethyl \\ 
 3 & propane & propyl \\ 
 4 & butane & butyl \\ 
 5 & pentane & pentyl \\ 
 6 & hexane & hexyl \\ 
 7 & heptane & heptyl \\ 
 8 & octane & octyl \\ 
 9 & nonane & nonyl \\ 
 10 & decane & decyl \\ 
 \hline
\end{tabular}
\end{center}

\noindent{\large \textbf{alkane}} - saturated hydrocarbon. It only has C and H and all bonds are single bonds. Example: \par
\vspace{3mm}
\centerline{
\chemfig{CH_3%6 these numbers are the carbon positions in the chain for ease of reading
-[:30]%5
-[:-30]%4
-[:30]%3
-[:-30]%2
-[:30]CH_3}%1
}

\vspace{5mm}
\noindent{\large \textbf{alkene}} - unsaturated hydrocarbon. It only has C and H there is at least one double bond. Example: \par
\vspace{3mm}
\centerline{
\chemfig{CH_3%6 these numbers are the carbon positions in the chain for ease of reading
-[:30]%5
-[:-30]%4
-[:30]%3
=[:-30]%2
-[:30]CH_3}%1
}

\vspace{5mm}
\noindent{\large \textbf{alkyl}} - an alkyl substituent is an alkane missing one hydrogen. A substituent is an atom or group of atoms substituted in place of a hydrogen atom on the parent chain of a hydrocarbon. Typically an alkyl is a part of a larger molecule. The smallest alkyl group is methyl, with the formula CH3�. Example: \par
\vspace{3mm}
\centerline{
\chemfig{-C(-[:90]H)(-[:-90]H)-H}
}

\vspace{5mm}
\noindent{\large \textbf{R}} - used to designate a generic (unspecified) alkyl group in a structural formula.

\vspace{5mm}
\noindent{\large \textbf{ether}} \par
\vspace{2mm}
\centerline{
\chemfig{
  R-[::30]O-[::-60]R'}
}

\vspace{5mm}
\noindent{\large \textbf{carobnyl}} - a carbonyl group is a functional group composed of a carbon atom double-bonded to an oxygen atom: C=O. A compound containing a carbonyl group is often referred to as a carbonyl compound. Example carbonyl compound where A and B can be anything: \par
\vspace{2mm}
\centerline{
\chemfig{
  A-[::30]O(=[:90]O)-[::-60]B}
}

\vspace{8mm}
\noindent{\large \textbf{aldehyde}} \par
\vspace{2mm}
\centerline{
\chemfig{
  R-[:30]C(=[:90]O)-[:-30]H}
}

\vspace{8mm}
\noindent{\large \textbf{ketone}} \par
\vspace{2mm}
\centerline{
\chemfig{
  R_1-[:30]C(=[:90]O)-[:-30]R_2}
}

\vspace{8mm}
\noindent{\large \textbf{ester}} \par
\vspace{2mm}
\centerline{
\chemfig{
  R-[:30]C(=[:90]O)-[:-30]OR'}
}

\pagebreak
\vspace{8mm}
\noindent{\large \textbf{carboxylic acid}} \par
\vspace{2mm}
\centerline{
\chemfig{
  R-[:30]C(=[:90]O)-[:-30]OH}
}
  
\vspace{8mm}
\noindent{\large \textbf{amino acid}} \par
\vspace{2mm}
\centerline{
\chemfig{H_2N-C(-[:90]H)(-[:-90]R)-[:30]C(=[:90]O)(-[:-30]OH)}
}

\vspace{8mm}
\noindent{\large \textbf{anhydride}} \par
\vspace{2mm}
\centerline{
\chemfig{
  -[:30]C(=[:90]O)-[:-30]O-[:30]C(=[:90]O)-[:-30]}
}

\vspace{8mm}
\noindent{\large \textbf{malonate}} \par
\vspace{2mm}
\centerline{
\chemfig{
  -[:-30]O-[:30]C(=[:90]O)-[:-30]-[:30]C(=[:90]O)-[:-30]O-[:30]}
}

\vspace{8mm}
\noindent{\large \textbf{alcohol}} - Any R with an OH, i.e., ROH \par
\vspace{2mm}
\centerline{
\chemname{\chemfig{ R-OH}}{primary} \qquad
\chemname{\chemfig{ R_1-C(-[:90]OH)(-[:-90]H)-R_2}}{secondary}  \qquad
\chemname{\chemfig{ R_1-C(-[:90]OH)(-[:-90]R_3)-R_2}}{tertiary}  \qquad
 }

\pagebreak
\vspace{8mm}
\noindent{\large \textbf{aryl}} - any functional group or substituent derived from an aromatic ring, be it phenyl, naphthyl, thienyl, indolyl, etc. 

\vspace{8mm}
\noindent{\large \textbf{cyclohexane}} \par
\vspace{2mm}
\centerline{
\chemfig{ *6(------)}
}

\vspace{8mm}
\noindent{\large \textbf{benzene}} \par
\vspace{2mm}
\centerline{
\chemfig{ *6(-=-=-=)}
}

\vspace{8mm}
\noindent{\large \textbf{phenyl}} \par
\vspace{2mm}
\centerline{
\chemfig{ R-*6(-=-=-=)}
}

\vspace{8mm}
\noindent{\large \textbf{phenol}} - simplest phenyl \par
\vspace{2mm}
\centerline{
\chemfig{ HO-*6(-=-=-=)}
}

\pagebreak
\vspace{8mm}
\noindent{\large \textbf{benzyl}} \par
\vspace{2mm}
\centerline{
\chemfig{*6(-=-(-C-[::60]R)=-=)}
}

\vspace{8mm}
\noindent{\large \textbf{Other prefixes}} \par
\begin{center}
\begin{tabular}{ |c|c|c| } 
 \hline
 Number & Prefix \\ 
 \hline
 1 & mono- \\ 
 2 & di- \\ 
 3 & tri- \\ 
 4 & tetra- \\ 
 5 & penta- \\ 
 6 & hexa- \\ 
 7 & hepta- \\ 
 8 & octa- \\ 
 9 & nona- \\ 
 10 & deca- \\ 
 \hline
\end{tabular}
\end{center}

\pagebreak
\noindent------------------------------------------------------------------------------------------

\vspace{5mm}
\chemfig{CH_3%6 these numbers are the carbon positions in the chain for ease of reading
-[::30]%5
-[::-60]%4
%(-[::-60]-[::60]CH_3)%ethyl
-[::60]%3
%(-[::30]CH_3)%methyl
%(-[::90]CH_3)%methyl
-[::-60]%2
-[::60]CH_3%1
}

\vspace{5mm}
\chemfig{CH_3%6 these numbers are the carbon positions in the chain for ease of reading
-[::30]%5
-[::-60]%4
(-[::-60]-[::60]CH_3)%ethyl
-[::60]%3
%(-[::30]CH_3)%methyl
%(-[::90]CH_3)%methyl
-[::-60]%2
-[::60]CH_3%1
}

\vspace{5mm}
\setatomsep{2em}
\chemfig{
  H_3C-[:72]{\color{blue}N}
    *5(-
      *6(-(={\color{red}O})-{\color{blue}N}(-CH_3)-(={\color{red}O})-{\color{blue}N}(-CH_3)-=)
    --{\color{blue}N}=-)}

\end{document}