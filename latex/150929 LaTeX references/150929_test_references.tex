\documentclass{article} % say 
\usepackage{tikz} 
\usepackage[super,sort&compress,comma]{natbib} 
\begin{document}
We are working on
 \begin{tikzpicture}
  \draw (-1.5,0) -- (1.5,0);
  \draw (0,-1.5) -- (0,1.5);
  \draw (0,0) circle [radius=1cm];
\end{tikzpicture}.

As discussed in Ref. \citenum{Ho2015}, the last few years have seen a steep increase in microfluidic publications involving 3D printing. This is indicative of growing interest in 3D printing for rapid prototyping of microfluidic devices in which devices are fabricated layer-by-layer directly from a 3D CAD design. A particularly promising 3D printing method for microfluidics is stereolithography (SLA) based on Digital Light Processing (DLP). In this approach a micromirror array is used to optically define the pattern for an individual layer by selective photopolymerization of a photo-sensitive resin. Successive layers of resin are exposed with appropriate optical patterns to fabricate an entire device. (REFS)

For successful 3D printing, the key feature of microfluidic devices is that they consist primarily of a series of small (micro) voids inside the polymerized material. These voids form a variety of structures including passive components\cite{Bhargava2014,Shallan2014,Au2014} (Bhargava 2014, Shallan 2014, Au 2014) such as flow channels, splitters, mixers, reaction chambers, and droplet generators, and active components such as valves (us and Au) and pumps.\cite{Au2015} Note that this emphasis on small voids is in direct contrast to many typical 3D printing applications in which external features or sparse structures are important. 

%%%REFERENCES%%%
\bibliography{references_jabref} %You need to replace "rsc" on this line with the name of your .bib file
\bibliographystyle{rsc} %the RSC's .bst file

\end{document}