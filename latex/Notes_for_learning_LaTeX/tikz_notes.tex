\documentclass{article}

\usepackage{tikz}
\usetikzlibrary{positioning, arrows.meta}
\usetikzlibrary{calc}

\usepackage{xcolor}
\usepackage[colorlinks = true,
            linkcolor = blue,
            urlcolor  = blue,
            citecolor = blue,
            anchorcolor = blue]{hyperref}

\title{Things I've Learned Working With Tikz}
\date{April 2016}
\author{Greg Nordin}

\begin{document}

\maketitle

\begin{enumerate}

    \item Don't start with complicated approaches that I don't understand. Instead, begin with the absolute simplest implementation that shows the fundamental properties I'm interested in. Use simple fixed coordinates--don't try to make the exploratory approach general or adjustable. Save that for later after the fundamentals have been demonstrated. 

    \item After showing the fundamental drawing element or visual aspects I want, incrementally increase the complexity/generality, compiling every step of the way so I don't spend unproductive time with unfound bugs confounding the process.

    \item When I google an issue for information, rapidly determine if it is a complicated solution. If so, reject it and look for something simpler. Don't spend time trying to figure it out because this can balloon into a massive time sink and misdirect subsequent efforts.

    \item Begin with simple documentation to learn a new feature, rather than looking at a Stackexchange question/answer that involves more complicated and likely unneeded elements.

    \item Review \href{http://cremeronline.com/LaTeX/minimaltikz.pdf}{minimaltikz.pdf}, which is especially good with
    	\begin{itemize}
		\item \verb!\draw!
		\begin{itemize}
			\item lines (single or multiple segments)
			\item decorators to specify line thickness, color, arrows, help lines, dashed, dotted, etc.
			\item line out and in angles
			\item rectangle, circle, arc
			\item plotting including axes
		\end{itemize}
		\item Filling area and setting borders
		\item \verb!\node!
		\begin{itemize}
			\item standalone nodes and nodes defined in a \verb!\draw! statement
			\item Adding text labels
			\item Locating labels relative to node coordinate
		\end{itemize}
	\end{itemize}

    \item Develop my own documentation and examples to remind myself what I have previously learned so I don't have to go through it all again. Examples:
	\begin{itemize}
		\item \verb!\draw!
		\item \verb!\node!
		\item \verb!\coordinate!
		\item \verb!\newcommand! - see https://en.wikibooks.org/wiki/LaTeX/Macros for how to use
		\item \verb!calcs! package
		\item \verb!.styles!
		\item \verb!positioning! package (still need to learn)
		\item Kinds of documents 
		\begin{itemize}
			\item Page with embedded figures: \\ \verb!\documentclass{article}!
			\item Standalone figure: \\ \verb!\documentclass[border=10pt, convert={density=300}]{standalone}!
		\end{itemize}
	\end{itemize}
    
\end{enumerate}


\end{document}