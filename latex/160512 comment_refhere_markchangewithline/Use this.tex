\documentclass{article}

% Try package changebar
% From http://tex.stackexchange.com/questions/175755/vertical-line-next-to-a-block-with-no-margins
\usepackage[color]{changebar}

% This provides the command, \ul{}, that supports line breaks in the underlined text
\usepackage{soul}

% From http://blog.rtwilson.com/how-to-add-simple-new-commands-to-latex-to-help-with-writing-papers/
\usepackage{xcolor}
\newcommand{\todo}[1] {\textbf{\textcolor{red}{#1}}}
\newcommand{\rh}{\todo{REF HERE }}

\begin{document}

Here is some text. \todo{and a to-do comment}. Some more text. \rh

\cbstart
Here is some more text.
\cbend

And more, and more, and more, and more....

\cbcolor{red}

\cbstart
In this paper we demonstrate that 3D printing with a Digital Light Processor stereolithographic (DLP-SLA) 3D printer can be used to create high density microfluidic devices with active components such as valves and pumps. 
Leveraging our previous work on optical formulation of inexpensive resins (RSC Adv. 5, 106621, 2015), we demonstrate valves with only 10\% of the volume of our original 3D printed valves (Biomicrofluidics 9, 016501, 2015), which were already the smallest that have been reported. 
\cbend

In this paper we demonstrate that 3D printing with a Digital Light Processor stereolithographic (DLP-SLA) 3D printer can be used to create high density microfluidic devices with active components such as valves and pumps. 
\cbstart Leveraging our previous work on optical formulation of inexpensive resins (RSC Adv. 5, 106621, 2015), we demonstrate valves with only 10\% of the volume of our original 3D printed valves (Biomicrofluidics 9, 016501, 2015), \cbend which were already the smallest that have been reported. 

Moreover, we show that inclusion of a thermal initiator in the resin formulation along with a post-print bake can dramatically improve the durability of 3D printed valves up to 1 million actuations. 
\cbstart Using two valves and a valve-like displacement chamber (DC), we also create compact 3D printed pumps. 
\cbend
With 5-phase actuation and a 15 ms phase interval, we obtain pump flow rates as high as 40 $\mu$L/min. 
We also characterize maximum pump back pressure (i.e., maximum pressure the pump can work against), maximum flow rate (flow rate when there is zero back pressure), and flow rate as a function of the height of the pump outlet. \underline{We further demonstrate combining 5 valves and one DC to create a 3-to-2 multiplexer with integrated pump. } 
\ul{In addition to serial multiplexing, we also show that the device can operate as a mixer.} 
Importantly, we illustrate the rapid fabrication and test cycles that 3D printing makes possible by implementing a new multiplexer design to improve mixing, and fabricate and test it within one day.


\end{document}