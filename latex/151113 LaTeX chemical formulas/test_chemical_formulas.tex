\documentclass{article}
\usepackage{mhchem}
\usepackage{chemfig}
\renewcommand*\printatom[1]{\ensuremath{\mathsf{#1}}}
\begin{document}

Using package mhchem we can write chemical formulas like this: 
\ce{SO4^2-},
\ce{Na2 SO4^2-},
\ce{PO4^3-}

\vspace{5mm}

From http://tex.stackexchange.com/questions/52722/can-you-make-chemical-structure-diagrams-in-latex

\vspace{5mm}

\setcrambond{2pt}{}{}
\chemfig{
  HO-[2,.5,2]?<[7,.7](-[2,.5]OH)-[,,,,line width=2.4pt](-[6,.5]OH)>[1,.7]
    (-[:-65,.7]O-[:65,.7]?[b](-[2,.7]CH_2OH)<[:-60,.707](-[6,.5]OH)
      -[,,,,line width=2.4pt](-[2,.5,,2]HO)>[:60,.707](-[6,.5]CH_2OH)-[:162,.9]O?[b])
  -[3,.7]O-[4]?(-[2,.3]-[3,.5]HO)}

\vspace{5mm}

\setatomsep{2em}
\chemfig{
  H_3C-[:72]{\color{blue}N}
    *5(-
      *6(-(={\color{red}O})-{\color{blue}N}(-CH_3)-(={\color{red}O})-{\color{blue}N}(-CH_3)-=)
    --{\color{blue}N}=-)}

\end{document}